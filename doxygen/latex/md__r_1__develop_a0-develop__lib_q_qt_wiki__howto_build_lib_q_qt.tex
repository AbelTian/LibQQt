这块要说的特别多。为什么呢?因为这块的工作很多,必须设计合理才会容易编译。 这里必须提一下\+Multi Link技术,是这个技术帮助使用者,利用qmake这款工程管理工具来进行多个增删link-\/library。全称\+Multi-\/link technology,“多链接”技术,专门解决这些个app链接library出现的手工解决太多困难的问题。 先说说:

\section*{我遇到的问题}


\begin{DoxyEnumerate}
\item 我的工程在自己电脑上编译很顺利,到了别的电脑上却需要经过非常大量的配置才能完成编译。
\item 我的工程我自己开发修改很容易,编译也很容易,可是和别人一起合作开发之后,修改却变得很困难,需要处理过多的冲突。
\item 工程配置工作竟然那么巨大数量,几天的功夫连一个工程都配置不完。
\item 工程管理概念太复杂,我不爱看。
\end{DoxyEnumerate}

这些我想读者都遇到过,我也遇到过,总是被这种路径那种路径困扰,总是被这个编译选项那个编译设置困扰。 说说我是怎么解决的: \section*{解决方法}

这里不做每一条的解说,这里做一个可行性的设计方案。 Qt是跨平台的library解决方案,虽然存在某些bug和不足,但不能掩盖它的确在跨平台解决方案中独占头席,无出其右。


\begin{DoxyItemize}
\item 开始一个工程之前,设置\+Qt Creator的默认编译目录,在设置-\/构建-\/概要 里面,基于\+Qt4的版本没有这个设置,那个建议不要使用。
\begin{DoxyItemize}
\item \%\{JS\+: Util.\+asciify(\char`\"{}/your/local/path/to/build/root/\%\{\+Current\+Project\+:\+Name\}/\%\{\+Current\+Kit\+:\+File\+System\+Name\}/\%\{\+Qt\+:\+Version\}/\%\{\+Current\+Build\+:\+Name\}\char`\"{})\}
\item \%\{Current\+Project\+:Name\} 代表当前工程的名字
\item \%\{Current\+Kit\+:File\+System\+Name\} 目标系统名称,这个在设置\+Qt Kit的时候,会要求设置,这里也要求设置,设置为和qqt\+\_\+qkit.\+pri里面定义的不同的\+S\+Y\+S\+N\+A\+M\+E变量的值相同,生成的时候要用,链接的时候要用,发布拷贝库的时候也要用,还是设置成一样的吧。
\item \%\{Qt\+:Version\} Qt的版本号
\item \%\{Current\+Build\+:Name\} Debug还是\+Release
\end{DoxyItemize}
\end{DoxyItemize}

这样出来的结果是 /..buildstation/\+Q\+Qt/5.\+9.\+2/mac\+O\+S/\+Debug 我的编译目标就在这里,拷贝\+S\+D\+K的时候也在这里拷贝,而且拷贝的目的地址也是这个样子。 这样管理起来是不是很费劲,手动拷贝得走进多少目录? 的确是这样的多层多个目录,\+M\+L\+M\+A技术就是解决这个问题。


\begin{DoxyItemize}
\item 设置\+Q\+K\+I\+T环境变量,\+Qt的build配置当中提供设置环境变量的位置 ~\newline
 有的读者疑惑,这里的设置也好多啊。的确,的确好多,但是这个多,根本不多,如果把那些拷贝不同目标系统的librayr的工作全部手动操作,那样加起来才算多,基本上开发一个平台的工作,要手动复制\+N个平台份。 我一般只设置\+Debug里的,\+Release里的一般就不设置了。 \%\{Current\+Kit\+:File\+System\+Name\} 我表示\+Qt Creator的这个变量,在qmake pro文件中还无法获取。等着吧,等着\+Qt开发组有空了把这个变量加入qmake,连\+Q\+K\+I\+T这个环境变量也不用设置了。
\item 运行qmake,编译终端输出所需的变量信息,根据提示,还是再\+Qt Creator中打开相应的app\+\_\+configure.\+pri进行修改,添加这几个变量。这个变量有三个。
\begin{DoxyItemize}
\item Q\+Q\+T\+\_\+\+B\+U\+I\+L\+D\+\_\+\+R\+O\+OT 这个代表工程编译的根路径,基本上所有的\+Qt工程编译都在这个根路径当中,\+Q\+Qt也不例外。
\item Q\+Q\+T\+\_\+\+S\+D\+K\+\_\+\+R\+O\+OT 这个代表\+Q\+Qt library将要导出\+S\+D\+K的根目录位置,基本上所有的library S\+D\+K都会导出到那里,这样做是有好处的,你想想,所有的library都在一个地方被链接,是不是管理起来更容易。在我的电脑上所有不论是哪家开发的library都放在了b0-\/product/lib里面。\+Q\+Qt依赖\+M\+L\+M\+A技术自动拷贝过去。
\item A\+P\+P\+\_\+\+D\+E\+P\+L\+O\+Y\+\_\+\+R\+O\+OT 这个变量可选
\end{DoxyItemize}
\end{DoxyItemize}

M\+L\+M\+A技术会制作一套pri,能够自动生成一些lib\+\_\+$\ast$.pri和app\+\_\+$\ast$.pri,用户对其进行完成,以制作出library和app。\+M\+L\+M\+A会自动的进行多个library编译、多个app编译链接,library发布\+S\+D\+K到指定位置,发布app到指定位置等工作。提供很多易用函数,方便用户使用qmake工作。

现在版本的\+M\+L\+M\+A技术和\+Q\+Qt捆绑在一起,还未实现分离。经过以上设置,编译出\+App应该没有多大问题。\+Q\+Qt工程我已经制作加调试了两年有余,运行质量还是有保证的。配套完善的\+M\+L\+M\+A技术,在跨平台开发的场合,更加容易使用。

在\+User新开发的\+App当中使用\+Q\+Qt,不需要拷贝pri,只要include app\+\_\+configure.\+pri就可以了,如果有需要,还可以include app\+\_\+deploy.\+pri完成app发布工作。

另外我再单独写一篇文章介绍\+Multi Link技术,说说这个技术的特点和包含有哪些功能丰富的pri。

\#注释 ~\newline
 链接方法在v2.1.\+6就更新了,细节请参照博客。(参照首页链接) 