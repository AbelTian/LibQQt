Multi Link 技术,这个技术帮助使用者利用qmake这款工程管理工具来进行多个增删link-\/library。全称\+Multi-\/link technology,“多链接”技术,专门解决这些个app链接library出现的手工解决太多困难的问题。

现在的\+Multi Link和\+Lib\+Q\+Qt绑定在一起,还未分离。在这里介绍现在的样子和将会实现的功能。 ~\newline
 Multi-\/link Technology Project Link\+: \href{https://gitee.com/drabel/multi-link}{\tt https\+://gitee.\+com/drabel/multi-\/link}

\section*{文件组成和功能介绍}


\begin{DoxyEnumerate}
\item qqt\+\_\+version.\+pri 用于更改\+Library的版本信息。只需要更改这一个地方就可以了
\item qqt\+\_\+qkit.\+pri 用于支持目标平台信息,读取环境变量\+Q\+K\+I\+T设置\+S\+Y\+S\+N\+A\+M\+E。
\item qqt\+\_\+function.\+pri 提供丰富的操作函数,分为两类两种,一类获取命令字符串,一类执行命令,一种条件函数,一种求解函数。
\begin{DoxyItemize}
\item mkdir system\+\_\+errcode read\+\_\+ini write\+\_\+ini empty\+\_\+file write\+\_\+line copy\+\_\+dir user\+\_\+home user\+\_\+config\+\_\+path
\item 相应的get函数
\item 支持windows 和 $\ast$nix。
\end{DoxyItemize}
\item qqt\+\_\+header.\+pri 包含qqt的特性开关,在这里决定library的大小和内部函数数目。
\begin{DoxyItemize}
\item Q\+Q\+T\+\_\+\+S\+T\+A\+T\+I\+C\+\_\+\+L\+I\+B\+R\+A\+RY 宏的开关判断也在这里。在linux等posix平台,这种区分动态静态的宏意义不是很大,一般不定义就可以。在windows平台要求很强烈,在\+Q\+Qt中,windows平台,编译静态库,无论编译还是链接,\+Q\+Q\+T\+\_\+\+S\+T\+A\+T\+I\+C\+\_\+\+L\+I\+B\+R\+A\+R\+Y都是需要的。动态库,编译时需要\+Q\+Q\+T\+\_\+\+L\+I\+B\+R\+A\+R\+Y宏,链接时不需要宏。在qqt.\+pro和link\+\_\+qqt.\+pri当中自动做了适配。
\end{DoxyItemize}
\item qqt\+\_\+source.\+pri 包含所有对应qqt的特性的源文件和头文件,\+H\+E\+A\+D\+E\+R\+S+= S\+O\+U\+R\+C\+E\+S+=全在这里。
\item qqt\+\_\+3rdparty.\+pri 包含所有的第三方源代码。对于\+G\+P\+L,第三方的还是单独拿出来。找问题和开关特性都比较清晰。
\item qqt\+\_\+install.\+pri 这里负责发布\+Q\+Qt S\+D\+K,所有的函数都在这里,但是发布工作在qqt\+\_\+library.\+pri里,编译\+Q\+Qt时不发生任何发布sdk的问题,只有在app当中pre link的时候才会create sdk,link from sdk。
\item qqt\+\_\+library.\+pri qqt的sdk发布工作,app链接\+Q\+Qt工作,要求用户进行\+B\+U\+I\+L\+D-\/\+S\+D\+K-\/\+D\+E\+P\+L\+OY R\+O\+O\+T路径指定也发生在这里。它会自动生成一个pri,给用户指定这三个变量。
\item qqt\+\_\+library.\+pri 包含以上这几个app需要的pri,这个文件现在只能放在library目录里。
\item app\+\_\+configure.\+pri app配置文件,这个一般是公共配置文件,里面包含了link\+\_\+qqt\+\_\+library.\+pri。更改意义不大,将来可能会被生成,而不是开始就存在。
\item app\+\_\+deploy.\+pri app发布文件,app发布到指定位置的工作就在这里实现,这个文件将来也可能被生成而不是开始就存在。
\end{DoxyEnumerate}

\section*{将要实现}


\begin{DoxyItemize}
\item qqt\+\_\+header.\+pri qqt\+\_\+source.\+pri 两个pri将会自动生成lib\+\_\+header.\+pri,lib\+\_\+source.\+pri,用户实现这两个pri,填充和自己的library有关的头文件和源文件,模块开关等信息。
\item qqt\+\_\+library.\+pri 将会自动生成app\+\_\+configure.\+pri用于用户配置
\item app\+\_\+deploy.\+pri 将会更名为qqt\+\_\+deploy.\+pri,用来完成支援app发布所需的功能。
\item qqt\+\_\+install.\+pri 将会负责和sdk发布相关的工作。
\item 这些pri将会全部转移到\+M\+L\+M\+A技术仓库,用户可以独立将其下载、配置,在一个地方完成多个lib和app工程的部署。
\begin{DoxyItemize}
\item 初步规划用户要clone到library目录。
\end{DoxyItemize}
\end{DoxyItemize}

相信在\+Multi Link技术的帮助下,用户开发\+Library和\+Application肯定会如虎添翼,节省巨大的工时和精力。(现在仅仅支持\+Lib\+Q\+Qt使用)

\section*{Multi-\/link技术完成}

千等万等,\+Multi-\/link技术终于现出原形了。 ~\newline
修复了发布\+Q\+Qt S\+D\+K无处安置的问题。 ~\newline

\begin{DoxyEnumerate}
\item include(.../multi-\/link/multi-\/link/add\+\_\+base\+\_\+manager.pri) ~\newline
 -\/ 这里是多链接技术一切的开始。 ~\newline
2. 提供函数 (这里只展示一部分)
\begin{DoxyItemize}
\item add\+\_\+sdk() (lib工程用) ~\newline
 -\/ add\+\_\+version() ~\newline
 -\/ add\+\_\+deploy() ~\newline
 -\/ add\+\_\+deploy\+\_\+library() ~\newline
 -\/ add\+\_\+deploy\+\_\+config() ~\newline
 -\/ add\+\_\+include() ~\newline
 -\/ add\+\_\+headers() ~\newline
 -\/ add\+\_\+library() ~\newline
 -\/ add\+\_\+defines() ~\newline
 -\/ add\+\_\+language() ~\newline
 -\/ add\+\_\+zh\+\_\+\+C\+N\+\_\+en\+\_\+\+U\+S() ~\newline
3. 彻底的脱离了lib\+Q\+Qt,但是提供对lib\+Q\+Qt的链接支持,和对其他\+Library的链接支持一模一样。 ~\newline
4. 有\+Multi-\/link帮助,用户可以任意的在app和lib之间设计链接关系了。 ~\newline
4. Q\+Qt提供的强大功能,并没有因为\+Multi-\/link的升级而改变,依然强大。 
\end{DoxyItemize}
\end{DoxyEnumerate}