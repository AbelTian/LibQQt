Lib\+Q\+Qt的版本演变。 ~\newline
 \tabulinesep=1mm
\begin{longtabu} spread 0pt [c]{*{6}{|X[-1]}|}
\hline
\rowcolor{\tableheadbgcolor}\textbf{ 命名  }&\textbf{ 时间  }&\textbf{ Lib\+Q\+Qt库  }&\multicolumn{3}{p{(\linewidth-\tabcolsep*6-\arrayrulewidth*4)*3/6}|}{\cellcolor{\tableheadbgcolor}\textbf{ Multi-\/link技术   }}\\\cline{1-6}
\endfirsthead
\hline
\endfoot
\hline
\rowcolor{\tableheadbgcolor}\textbf{ 命名  }&\textbf{ 时间  }&\textbf{ Lib\+Q\+Qt库  }&\multicolumn{3}{p{(\linewidth-\tabcolsep*6-\arrayrulewidth*4)*3/6}|}{\cellcolor{\tableheadbgcolor}\textbf{ Multi-\/link技术   }}\\\cline{1-6}
\endhead
犰狳  &2017年1月1日  &v1.\+0  &-\/  &-\/  &Lib\+Q\+Qt发布   \\\cline{1-6}
猎豹  &2017年12月1日  &v2.\+0  &v1.\+0  &R2  &Lib\+Q\+Qt2.\+0发布,基础功能和精美控件对嵌入式开发有优秀的表现。~\newline
 Multi-\/link技术发布,这时只支持链接\+Q\+Qt。~\newline
 R2适合只需要\+Lib\+Q\+Qt就可以完成应用的开发场景。   \\\cline{1-6}
海豚  &2018年6月19日  &v3.\+0  &v2.\+1  &R3 (master)  &Lib\+Q\+Qt3.\+0发布,主要对桌面开发场景进行了梳理,添加了一系列桌面上喜闻乐见的功能。 ~\newline
 Multi-\/link技术更新到2.1,提供对多种\+Library的链接支持,包括\+Q\+Qt。~\newline
 R3适合即需要\+Lib\+Q\+Qt的基础框架功能,又需要扩展像\+Open\+CV O\+SG O\+G\+R\+E等流行库的开发场景。   \\\cline{1-6}
\end{longtabu}


注释: ~\newline
$\ast$\+R分支是\+Lib\+Q\+Qt第\+N代的长期保持分支。\+R代\+V版本全都在上面。$\ast$ ~\newline
$\ast$\+R分支有利于保持各代长期bug最少。$\ast$ ~\newline
$\ast$master分支是最新的\+R分支。$\ast$ ~\newline


\subsection*{v2.\+2.\+1  changelog/v2.\+2.\+1.\+md \char`\"{}详情\char`\"{}}

\subsection*{v2.\+2  changelog/v2.\+2.\+md \char`\"{}详情\char`\"{}}