Lib\+Q\+Qt要求使用分别的独立的多个工程,管理\+Q\+Qt的编译和\+App的编译。
\begin{DoxyItemize}
\item 创建app工程,在\+Int\+Ins目录。不表。
\item 与\+Int\+Ins同目录,创建\+Lib\+Q\+Qt工程
\begin{DoxyItemize}
\item git clone \href{https://gitee.com/drabel/LibQt}{\tt https\+://gitee.\+com/drabel/\+Lib\+Qt} Lib\+Q\+Qt (当初新建工程的时候,手抖了一下,在mac下少写了一个Q..)
\end{DoxyItemize}
\end{DoxyItemize}

\section*{配置工程}

修改app的pro文件 如下:


\begin{DoxyCode}
\{c++\}
system(touch main.cpp)
include ($$\{PWD\}/../LibQQt/src/app\_configure.pri)
include ($$\{PWD\}/../LibQQt/src/app\_deploy.pri) (optional, 发布App用)
\end{DoxyCode}



\begin{DoxyItemize}
\item 添加\+Q\+K\+IT 环境变量,一般桌面上 会选择 Q\+K\+IT=W\+I\+N32 或者 Q\+K\+IT=L\+I\+N\+UX 或者\+Q\+K\+IT=mac\+O\+S等几种,桌面上一般就这三种,相应的还有64位的选择。
\item 运行qmake,根据错误提示和在link\+\_\+qqt\+\_\+library.\+pri下面生成的app\+\_\+configure.\+pri配置\+Q\+Q\+T\+\_\+\+B\+U\+I\+L\+D\+\_\+\+R\+O\+OT Q\+Q\+T\+\_\+\+S\+D\+K\+\_\+\+R\+O\+O\+T,如果包含了app\+\_\+deploy.\+pri还要配置\+A\+P\+P\+\_\+\+D\+E\+P\+L\+O\+Y\+\_\+\+R\+O\+O\+T。设他们=X\+X\+X目录。遵循qmake语法,这个应该不难。
\item 然后就开始编译,坐等编译完成。现在的master还算稳定,也就是v2.1.\+3,还算稳定,我在几个平台上都测试通过了,windows测试的比较少,但是一般也会通过,等详细测试了更新工程。
\end{DoxyItemize}

qqtframe2 demo,这个里面个空白的\+Main\+Window工程,用来向用户展示应该如何使用\+Lib\+Q\+Qt。 选择活动工程\+Int\+Ins 运行,就可以看到窗口了。

2018年01月10日19\+:09\+:14 ~\newline
exquisite是最近调试的最多的工程,通过这个工程看到的工程管理现在是最合适使用的。配合v2.1.\+6.\+0 