v1.\+0仅仅支持链接\+Lib\+Q\+Qt,并且依附在\+Lib\+Q\+Qt代码中。 ~\newline

\begin{DoxyItemize}
\item Qt Creator设置默认编译目录: ~\newline
\%\{JS\+: Util.\+asciify(\char`\"{}/your/local/path/to/build/root/\%\{\+Current\+Project\+:\+Name\}/\%\{\+Current\+Kit\+:\+File\+System\+Name\}/\%\{\+Qt\+:\+Version\}/\%\{\+Current\+Build\+:\+Name\}\char`\"{})\}
\end{DoxyItemize}

\section*{Multi-\/link v2.\+0}

在完成的\+Multi-\/link技术里,新的\+Q\+S\+Y\+S环境变量和\+Qt Kit的关系 ~\newline
Q\+K\+I\+T不再使用,而仅仅使用\+Q\+S\+Y\+S。 ~\newline
 \tabulinesep=1mm
\begin{longtabu} spread 0pt [c]{*{4}{|X[-1]}|}
\hline
\rowcolor{\tableheadbgcolor}\textbf{ Qt Kit  }&\multicolumn{3}{p{(\linewidth-\tabcolsep*4-\arrayrulewidth*2)*3/4}|}{\cellcolor{\tableheadbgcolor}\textbf{ Kit File System Nam   }}\\\cline{1-4}
\endfirsthead
\hline
\endfoot
\hline
\rowcolor{\tableheadbgcolor}\textbf{ Qt Kit  }&\multicolumn{3}{p{(\linewidth-\tabcolsep*4-\arrayrulewidth*2)*3/4}|}{\cellcolor{\tableheadbgcolor}\textbf{ Kit File System Nam   }}\\\cline{1-4}
\endhead
Windows 32bit  &Windows  &-\/  &Windows   \\\cline{1-4}
Windows 32bit  &Win32  &-\/  &Win32   \\\cline{1-4}
Windows 64bit  &Win64  &-\/  &Win64   \\\cline{1-4}
Linux 32bit  &Linux  &-\/  &Linux   \\\cline{1-4}
Linux 64bit  &Linux64  &-\/  &Linux64   \\\cline{1-4}
mac\+OS clang 64bit  &mac\+OS  &-\/  &mac\+OS   \\\cline{1-4}
Arm 32bit  &Arm32  &-\/  &Arm32   \\\cline{1-4}
Arm\+HF 32bit  &Armhf32  &-\/  &Armhf32   \\\cline{1-4}
Mips 32bit  &Mips32  &-\/  &Mips32   \\\cline{1-4}
Embedded 32bit  &Embedded  &-\/  &Embedded   \\\cline{1-4}
i\+OS clang  &i\+OS  &-\/  &i\+OS   \\\cline{1-4}
i\+OS Simulator  &i\+O\+S\+Simulator  &-\/  &i\+O\+S\+Simulator   \\\cline{1-4}
Android armeabi  &\+Android  &-\/  &\+Android   \\\cline{1-4}
Android x86  &\+Android\+X86  &-\/  &\+Android\+X86   \\\cline{1-4}
\end{longtabu}


\paragraph*{使用场景截图}

 ~\newline
 \paragraph*{Multi-\/link技术能够达到的管理能力}

App和\+Lib的源代码,一直处于编写之中。 ~\newline
App和\+Lib的目标,一直从\+Build位置,持续发布到\+Deploy位置和\+S\+D\+K位置。 ~\newline
用户再也不必为了管理生成目标、发布目标和链接而劳费手劲。 ~\newline
在2008年的时候还没有这个技术,2018年,这个技术终于变成了现实。 ~\newline
现在,按照\+G\+P\+L发布, ~\newline
基于qmake。 ~\newline
 ~\newline
 \paragraph*{多链接技术创造的生产线}

Multi-\/link会一直处于\+App/\+Lib生产线的控制器地位。 ~\newline
Multi-\/link允许用户自行添加任何依赖项,我把一些常用的依赖项添加用pri放在了app-\/lib里, 而这些依赖项的\+S\+D\+K我保存在了百度网盘,以方便用户取用,用户只需要下载下来解压到自己的\+L\+I\+B\+\_\+\+S\+D\+K\+\_\+\+R\+O\+O\+T里。 ~\newline
百度网盘地址链接:https\+://pan.baidu.\+com/s/1\+F\+P\+Pk\+T\+Unk2\+X\+B\+L4rpn\+Zs\+A\+Gmw 密码:hotz ~\newline
S\+D\+K难免有不全,难免不能满足任何用户的需求,请用户自行补齐。 利用\+Multi-\/link技术的添加\+Library模板很容易的。  ~\newline
 \paragraph*{多链接技术使用注意}


\begin{DoxyEnumerate}
\item Multi-\/link 2.\+0已经不强调 build的编译路径设置。
\item 也不强调对\+Qt Creator File\+System\+Name的设置。也就是不要求用户按照过去的要求设置\+Creator的默认编译路径。
\item Multi-\/link可以独立于任何\+Library App工程使用,建议作为submodule使用,能够可靠保证移植性。
\item 但是\+Q\+S\+Y\+S\+\_\+\+S\+T\+D\+\_\+\+D\+I\+R还是有用的,输出\+S\+D\+K和\+Deploy\+A\+P\+P的时候使用。并且,\+Q\+A\+P\+P\+\_\+\+S\+T\+D\+\_\+\+D\+I\+R和这个\+S\+D\+K的路径还不一样,\+App发布用的有个\+Debug和\+Release的区分,\+S\+D\+K的没有(区分),在一起。
\end{DoxyEnumerate}

\paragraph*{多链接技术已经支持的\+Library}

并且多链接技术已经提供了产品运行器和\+S\+D\+K链接文件编写辅助工具。 ~\newline
对\+Library的支持会持续更新。 ~\newline
用户也可以把使用的方便的add\+\_\+library\+\_\+xxx.\+pri发送给我到tianduanrui@163.\+com,我会把它加入app-\/lib族。 ~\newline
 ~\newline
 \paragraph*{多链接技术配置工具}

 ~\newline


\section*{Multi-\/link v2.\+1}


\begin{DoxyItemize}
\item v2.\+1默认不再链接\+Q\+Qt。 ~\newline
-\/ 建议用户把multi-\/link作为子模块clone下来。 ~\newline
-\/ 用户在include(multi-\/link/add\+\_\+base\+\_\+manager.\+pri)以后, ~\newline
 -\/ add\+\_\+dependent\+\_\+manager(\+Q\+Qt) 就完成了链接、跟随发布\+Q\+Qt。(内部自动判断\+Proj类型,\+App Proj才会发布,\+Lib Proj不会发布。) ~\newline
 -\/ add\+\_\+custom\+\_\+dependent\+\_\+manager(xxx)是在自定义目录加载add\+\_\+library\+\_\+xxx.\+pri,默认工程当前目录。 ~\newline
 -\/ add\+\_\+create\+\_\+dependent\+\_\+manager(xxx)是在自定义目录加载add\+\_\+library\+\_\+xxx.\+pri,不存在则使用模板创建这个pri,默认app-\/lib目录。 ~\newline
-\/ add\+\_\+deploy() 或者 add\+\_\+sdk()(Lib Proj) 是标准使用函数,任何工程都用他。
\item add\+\_\+version() 添加版本。 ~\newline
-\/ add\+\_\+lanauage() 添加语言。 ~\newline
-\/ 给用户提供了\+Add\+Lib\+Tool用于辅助用户从\+S\+D\+K\+R\+O\+O\+T直接获取add\+\_\+library\+\_\+xxx.\+pri文件。 ~\newline
-\/ Product\+Exec\+Tool用于帮助用户从产品库里运行程序,程序太多的时候,一个一个深入目录点击运行也是比较不容易。 ~\newline
 \section*{多链接技术使用说明}
\end{DoxyItemize}

看懂了v2的设计思路、原理以后,看看这里的多链接技术使用说明。 ~\newline
  ~\newline
 \section*{Multi-\/link v2在操作系统中的地位}

在操作系统里\+Multi-\/link的地位如下,用户说这是嵌入式操作系统的结构,其实这也是桌面操作系统的结构。 ~\newline
Multi-\/link 2主要应用于\+App和\+App Framework层的多关系链接工作。 在这种二进制不兼容的多种多样的系统当中,这种链接方式非常有用。 ~\newline
在\+Library层,通常比较现代的工程管理工具为\+C\+Make,而在\+App层比较流行的工程管理工具为qmake,跟随\+Qt发布的qmake。 ~\newline
  ~\newline
 \href{.}{\tt 返回} 