\subsection*{项目介绍}

为应用程序和链接库工程开发的,实现链接、发布依赖\+Library,发布\+S\+D\+K,发布应用,发布语言、配置等工程管理功能的多链接技术。 ~\newline
Multi-\/link技术使用众多的pri进行函数定义,提供给用户丰富的\+App/\+Lib生产线操作函数,省却手动拷贝\+App、\+Lib、依赖令手痛的问题。 ~\newline
我编写的\+Multi-\/link技术使用内置支持\+Library的方式支持众多的\+Library,方便共享对\+Library的支持,并且方便准确及时地同步到工程中进行使用,基本上编写一次,便不必再修改。 ~\newline
用户有使用方便的\+Library可以给我发邮件,tianduanrui@163.\+com.把add\+\_\+library\+\_\+\+X\+X\+X.\+pri发给我。我会把它提交到\+Multi-\/link工程里。 ~\newline
Multi-\/link技术位于全自动化构建技术的最关键的位置,即源代码工程全自动化构建、生产。

\subsection*{功能清单}


\begin{DoxyEnumerate}
\item 跨windows、mac\+O\+S、linux(笔者使用ubuntu kylin x64)三个平台,功能已经全面完成。
\item 这些功能,就是在qmake阶段设置好所有的用户过去需要手动做的工作,通过\+Q\+M\+A\+K\+E\+\_\+\+P\+O\+S\+T\+\_\+\+L\+I\+N\+K自动完成。
\item add\+\_\+function.\+pri add\+\_\+project.\+pri里面提供了丰富的基本功能,用户可以用其扩展技术外功能。
\item add\+\_\+deploy() 发布app到app发布目录 用于app工程。
\item add\+\_\+sdk() 发布sdk到sdk目录 用于library工程。
\item add\+\_\+dependent\+\_\+manager() 为工程添加依赖的\+S\+D\+K,它会到\+S\+D\+K目录查找具体的\+S\+D\+K,输入参数,sdk包名,sdk包内子模块名。比如:\+Qt, Widgets.\+这里吆告诉读者,把\+Qt S\+D\+K放到\+S\+D\+K目录里也有效果。通常这一个函数链接库的\+S\+D\+K就够用了,里面包含了add\+\_\+include() 包含头文件路径 add\+\_\+library() 链接库 add\+\_\+defines() 添加library的宏定义 add\+\_\+deploy\+\_\+library() 把库跟随app发布到app发布目录。
\item add\+\_\+deploy\+\_\+config() 将指定路径的配置文件发布到build路径和product发布路径。
\item add\+\_\+icons() 为应用程序添加logo,尤其windows和mac\+O\+S下。
\item add\+\_\+language() 为应用程序添加翻译文件,自动添加翻译文件,用户只需要找到文件翻译下就可以了。
\end{DoxyEnumerate}
\begin{DoxyEnumerate}
\item add\+\_\+version() 为应用程序添加版本信息。
\end{DoxyEnumerate}

\subsection*{提供的工具}

经过发布的\+App直接点击就可以运行,$\ast$大的省去了用户手动发布\+App的劳烦过程。 ~\newline
$\ast$\+Multi-\/link提供\+Product\+Exec\+Tool,可以对产品集中查看、调用运行。$\ast$ ~\newline
$\ast$\+Multi-\/link提供\+Add\+Library\+Tool,方便用户通过准备好的\+S\+D\+K自动生成add\+\_\+library\+\_\+xxx.pri链接环。$\ast$ ~\newline
$\ast$\+Multi-\/link提供\+Add\+Library\+Tool-\/\+Multiple,可以同时对多套\+S\+D\+K进行生成链接环。$\ast$ ~\newline
$\ast$\+Multi-\/link提供\+Multi-\/link\+Config\+Tool,方便用户配置\+Multi-\/link v2必需的三大路径,build/sdk/deploy root。$\ast$ ~\newline
$\ast$\+Multi-\/link提供\+Sdk\+List\+Tool,方便用户查看已经准备好的\+S\+D\+K在各个平台准备情况的表格。$\ast$ ~\newline
 \subsection*{软件架构}

\href{Multi-linkFunctionList.xlsx}{\tt 多链接技术的工程结构.\+xlsx} ~\newline
由于\+Qt第四代编译比较困难,\+Qt4内置的qmake版本2.01a版本太低,对函数的支持不足,对嵌套函数的支持也不足, ~\newline
所以,\+Multi-\/link2.0不支持\+Qt4。 ~\newline
Multi-\/link1.\+0绑定\+Q\+Qt,也不会继续开发与\+Q\+Qt脱离的纯粹使用pri的版本,\+Qt4 qmake版本太低,不便于开发。 ~\newline
 \subsection*{安装教程}


\begin{DoxyEnumerate}
\item 在用户主目录/.qmake/app\+\_\+configure.\+pri里面配置三个变量(\+Only Once) ~\newline
 -\/ L\+I\+B\+\_\+\+S\+D\+K\+\_\+\+R\+O\+OT = /home/abel/\+Develop/b1-\/sdk
\begin{DoxyItemize}
\item A\+P\+P\+\_\+\+B\+U\+I\+L\+D\+\_\+\+R\+O\+OT = /home/abel/\+Develop/c0-\/buildstation
\item A\+P\+P\+\_\+\+D\+E\+P\+L\+O\+Y\+\_\+\+R\+O\+OT = /home/abel/\+Develop/b0-\/product
\item 可以编译运行\+Multi-\/link\+Config\+Tool,实现一次图形化的配置,配置好了还会兼容\+Multi-\/link 1.\+0.
\item 配置一次就可以了,\+Multi-\/link提供的其他工具就都可以用了。 ~\newline
2. 在project build configure页面配置构建环境变量,\+Q\+S\+YS=Windows等指示平台变量(参见add\+\_\+platform.\+pri)。
\item 这个是每次每个build都需要配置的,这个有\+Qt Creator的开发历史原因。 ~\newline
 \subsection*{使用说明}
\end{DoxyItemize}
\end{DoxyEnumerate}
\begin{DoxyEnumerate}
\item 一个可以拷贝multi-\/link到自己工程目录,
\begin{DoxyItemize}
\item 一个可以clone multi-\/link到公共位置
\item 一个可以clone multi-\/link到工程目录作为submodule。这个是推荐方式,我对链接库们的支持容易使用到自己的工程里,我一般使用这个方式。
\end{DoxyItemize}
\item include (.../multi-\/link/add\+\_\+base\+\_\+manager.pri) ~\newline
4. add\+\_\+version() add\+\_\+deploy() add\+\_\+dependent\+\_\+manager(\+Q\+Qt) add\+\_\+dependent\+\_\+manager(\+X\+X\+X\+Lib) ...
\item 如果希望添加自定义模块,如果希望添加自己使用的其他的app-\/lib没支持的库,
\begin{DoxyItemize}
\item 那么从multi-\/link/app-\/lib里拷贝add\+\_\+custom\+\_\+manager.pri到工程目录(optional,\+Multi-\/link 1.\+0)。 ~\newline
 -\/ 使用\+Add\+Library\+Tool写自定义的add\+\_\+library\+\_\+\+X\+X\+X.\+pri(\+Multi-\/link 2.\+0),然后拷贝这个pri到工程目录,或者到\+Multi-\/link的app-\/lib目录,使用add\+\_\+custom\+\_\+dependent\+\_\+manager(X\+XX)/add\+\_\+dependent\+\_\+manager(X\+XX)调用. ~\newline
 详细使用说明 ~\newline

\end{DoxyItemize}
\end{DoxyEnumerate}

\subsection*{使用截图}

屏幕截图 ~\newline


\subsection*{约束}


\begin{DoxyEnumerate}
\item 源代码目录里multi-\/link目录有必要和代码目录平级。不应当把multi-\/link文件夹放到src目录里。 ~\newline
2. mac\+O\+S下,一切被依赖的\+Library不可以和依赖者\+App或者\+Library共同编译。免于触发first-\/time bug。
\begin{DoxyItemize}
\item 已经修复。修改了搬运软件的时机。 ~\newline
3. mac\+O\+S下,工程的\+C\+O\+N\+F\+I\+G,从+lib\+\_\+bundle变化到-\/lib\+\_\+bundle,需要手动删除工程编译目录里的所有framework。
\item cp命令的行为,我至今不明确。
\end{DoxyItemize}
\end{DoxyEnumerate}

\subsection*{总结}

看起来挺巨大的?工程量的确不小。 ~\newline
初学者注意看: ~\newline
1. 第一次使用,拿到\+Multi-\/link技术的仓库,使用\+Multi-\/link.pro编译\+Multi-\/link\+Config\+Tool,配置三个主要路径。
\begin{DoxyItemize}
\item 在这台电脑上,\+Multi-\/link Technology(qmake)开始臣服于你。 ~\newline
2. 顺便配置好\+Qt Creator吧,这两步只需要配置一次,这一步还是optional,配置下比较美观,反正就一次。 ~\newline
 -\/ 菜单栏-\/工具-\/选项-\/构建和运行
\begin{DoxyItemize}
\item 概要,设置你喜欢的,我建议的默认编译路径。
\item 构建套件,对每个\+Kit设置你喜欢,我建议的\+File system name。\+S\+D\+K路径、\+Deploy路径下面,肯定是使用我的建议的\+File system name。 ~\newline
 初学者,可以拿依赖\+Multi-\/link技术的\+L\+I\+B工程和\+A\+P\+P工程。 ~\newline
1. 用\+Qt Creator打开工程.\+pro。
\end{DoxyItemize}
\item 选择几个目标configure project。(.pro.\+user)。这是最开始。
\item 左侧菜单-\/项目-\/\+Build\&Run-\/\+Build-\/构建设置-\/构建环境-\/详情 ~\newline
 -\/ 批量编辑,添加\+Q\+S\+YS=Windows/mac\+O\+S/\+Linux64/\+Android/\+Armhf32/i\+O\+S等选一,target\+Name就是上边我建议的\+File system name,在add\+\_\+platform.\+pri里边能找到。 ~\newline
2. 用\+Qt Creator编译源代码工程
\item 工程树-\/工程名-\/鼠标右键
\begin{DoxyItemize}
\item 执行qmake。 =qmake
\item 构建。 =make
\end{DoxyItemize}
\item 到\+S\+D\+K仓,产品库看看吧,有产品了。
\begin{DoxyItemize}
\item S\+D\+K\+List\+Tool,可以帮助查看\+S\+D\+K仓。
\item Product\+Exec\+Tool,可以帮助运行产品库里的软件。
\end{DoxyItemize}
\end{DoxyItemize}

用\+Qt Creator编辑源代码工程 ~\newline
 -\/ 这个没什么可说的,按照常规使用\+Qt Creator编辑源代码的习惯编辑即可。 ~\newline
 -\/ 切记,使用\+Multi-\/link技术,只要使用其提供的qmake函数,千万不要去手动触碰\+S\+D\+K,甚至\+App。这是\+Multi-\/link技术的初衷。 ~\newline


\subsection*{联系我}

邮箱: \href{mailto:tianduarnui@163.com}{\tt tianduarnui@163.\+com} ~\newline
QQ\+: 2657635903 