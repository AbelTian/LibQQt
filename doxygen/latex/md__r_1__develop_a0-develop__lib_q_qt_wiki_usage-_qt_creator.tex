有一些同学认为安装\+Qt Creator就是安装\+Qt,不是的。

Qt Creator是一款使用qmake对用户\+Qt工程进行管理的\+I\+D\+E。

qmake是我们常说的generate make工具,理论上它独立于\+Qt,即独立于\+Qt Library。 ~\newline
qmake理论上独立于\+Qt,但是\+Qt设计者将其依附于\+Qt之中,用于帮助\+Qt Creator区分\+Qt的版本(目标跨平台)。 ~\newline
qmake跟随\+Qt发布,即跟随\+Qt Library发布,通常我们所说的\+Qt是\+Qt Library。 ~\newline
Qt版本来源于\+Qt Library版本,qmake的版本独立于\+Qt版本, Qt Creator版本独立于\+Qt版本。 我们一般只关注\+Qt版本和qmake所指向的\+Qt版本,它俩其实是一个概念。

Qt Creator包括对\+Qt Versions的支持,通过查找qmake实现,它支持多版本、多目标的\+Qt。 ~\newline
它包含对qmake、cmake的使用能力,也就是能管理工程源代码树。 它包含对make的使用能力,也就是能编译。 ~\newline
它包含对gdb、windbg等调试器的支持,也就是能调试。 ~\newline
另外, ~\newline
它包含对astyle、clang-\/format的支持,也就是能保存时格式化代码。 ~\newline
它包含对源码外编译(shadow build)的支持,也就是对\+Qt Creator设置的默认编译目录,指定了qmake和编译器工作的根位置。这里注意:\+Qt Creator可以根据自身的变量通配目录名,这些变量不是qmake的,它没有把这些变量完全传给qmake的pr\mbox{[}o,i,f\mbox{]}们。(\+Qt5) ~\newline
 \section*{安装目录}

Qt Creator通常跟随于\+Qt各个版本发布一个版本,用户选择新版本使用就好了,在安装目录\+Tools目录里。按个最新的\+Qt,找\+Tools目录就用它。 ~\newline
Qt 其他的版本,把安装目录里的\+Qt版本号目录,剪切到这个新版本的\+Qt安装根目录里,并排各版本号目录,\+Qt Versions那里随便选好了,你使用了哪个版本的\+Qt、针对了哪个目标,你自己知道。 ~\newline
Qt Doc目录,你在里边放哪个版本的\+Qt Doc在\+Qt Creator的help里就会出现哪个让你选,操作操作就知道了,除非版本间区别太大,一般放一个。 ~\newline
Qt Creator的配置目录,类\+Unix系统,在用户根目录/.config/\+Qt\+Project里,\+Windows下,在用户根目录/\+App\+Data/\+Roaming/\+Qt\+Project里。 ~\newline
 以上是对\+Qt Creator这款\+I\+D\+E的介绍。 ~\newline
 \href{.}{\tt 返回} 